\documentclass[a4paper]{article}

\usepackage[english]{babel}
\usepackage[utf8x]{inputenc}
\usepackage{amsmath}
\usepackage{graphicx}
\usepackage[left=1.25 in, right=1.25 in, top=1.25 in, bottom=1.25 in]{geometry}
\usepackage{hyperref}
\usepackage{gensymb}
\usepackage{bbold}


% Bibliography
\usepackage[numbers, compress]{natbib} % Bibliography - APA
\bibpunct{(}{)}{;}{a}{}{,}
\usepackage{lineno} % Line numbers
\def\linenumberfont{\normalfont\footnotesize\ttfamily}
\setlength\linenumbersep{0.2 in}

\usepackage{setspace}

\newcommand{\ignore}[1]{}

\begin{document}

\noindent
\textbf{Title}: You are where you eat:  Linking feral swine movement to resource use within and across pig populations

\bigskip

\noindent
\textbf{Running title}:

\bigskip

\noindent
\textbf{Authors}: Mark Q. Wilber$^1$, Sarah Chinn

\bigskip

\noindent
\textbf{Author affiliations}: \\

\bigskip

\noindent
\textbf{Corresponding author}:

\bigskip

\noindent
\textbf{Word count}:

\bigskip

\noindent
\textbf{Keywords}:

\newpage

\doublespacing
\linenumbers

\section*{Abstract}

\section*{Introduction}

Outline

Paragraph 1: 
	- Could start out pig specific.  The problems pigs cause etc.
	- Could start more general regarding how individual-movement is diverse, but being able to elucidate general patterns across scales can help understand resource use and aid management (sorry Hugh! I will think about how).
	- Why do we care about movement? The first step to understanding population dynamics.  If we can get a good sense of movement dynamics we can explore disease transmission, reproduction, damage, etc. Individual-level movement is a means to an end. 

Paragraph 2:
	- What do we know about pig movement (Harltey et al., Kay et al. 2017, Garza, Snow, Lewis, McClure).
	- With the exception of Hartley and Kay, these studies focused on movement at the large scale pig movement and and did not explicitly account for the dynamic pig movement\?
	- The disadvantage of these approaches is that they don't provide much to build upon.  In contrast, a dynamic model of pig movement and resource use will be applicable for population-level inferences. 
	-  Most movement studies have been focused on home-range size and factors than affect home range (Garza, and others).  While home range captures a fundamental aspect of feral swine biology, it misses critical patterns of movement \emph{within} a home range, which can have important implications for contact, reproduction, and resource use. 

\section*{Methods}

\subsection*{Data}

\subsubsection*{Feral swine movement data}

\subsubsection*{Covariates}

\subsection*{Movement models}

To understand how resources, etc. affected pig movement within and across populations, we used the modeling framework of \cite{Hanks2015} and \cite{Wilson2018}. Broadly, this framework uses a continuous-time, discrete-space movement model to make two types of inference: 1) how various biotic and abiotic covariates (i.e. "resources") affect pig movement rate and the direction of pig movement \citep{Hanks2015} and 2) the long-term probability of resource use for an organism in a given area, based on a dynamic model of individual movement \citep{Wilson2018}.  The key advantage of this approach for our purposes is that the continuous-time nature allows us to sync-up time scales across various populations, such that we are making inference regarding [resource selection] at consistent time-scales across different studies. 

Specifically, this approach can be broken into two distinct steps which we describe in detail below: fitting a continuous-time movement model to the movement data and using this continuous-time movement model to explore how resource selection affects pig movement.

\subsection*{Continuous-time movement models}

% While discrete-time movement models have a rich history of use in movement ecology [citations], continuous-time movement models have recently been gaining prominence \citep[e.g.][]{Johnson2008a,Hanks2015,Buderman2016,Hooten2017} [Brownian bridge stuff, etc.].  One advantage of continuous-time movement models is that they naturally handle movement data in which time steps are not equal, without having to resort to various state-space/missing data approaches that are often used when accounting for unequal time steps in discrete-time movement models. However, fitting continuous-time movement models 

To fit a continuous-time movement model to GPS data for a particular pig, we used basis function approach developed by \cite{Buderman2016}.  This is a phenomenological approach to modeling pig movement and proceeds by fitting linear model to pig movement data. [And Buderman 2018] developed a basis function approach for fitting continuous-time movement models to animal movement data that [advantages of approach].  




\subsection*{Analysis}



\section*{Results}
{}
\section*{Discussion}

\singlespacing
\bibliographystyle{/Users/mqwilber/Dropbox/Documents/Bibformats/mee.bst}
\bibliography{/Users/mqwilber/Dropbox/Documents/Bibfiles/Projects_and_Permits-feral_swine.bib}



%\section*{References}

\end{document}