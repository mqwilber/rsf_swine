%%%%%%%%%%%%%%%%%%%%%%%%%%%%%%%%%%%%%%%%%
% Beamer Presentation
% LaTeX Template
% Version 1.0 (10/11/12)
%
% This template has been downloaded from:
% http://www.LaTeXTemplates.com
%
% License:
% CC BY-NC-SA 3.0 (http://creativecommons.org/licenses/by-nc-sa/3.0/)
%
%%%%%%%%%%%%%%%%%%%%%%%%%%%%%%%%%%%%%%%%% 

%----------------------------------------------------------------------------------------
%   PACKAGES AND THEMES
%----------------------------------------------------------------------------------------

\documentclass[xcolor=dvipsnames]{beamer}
\usepackage{graphicx} % Allows including images
\usepackage{tikz}
\usepackage{pgfplots}
\usepackage{pgf}
\usepackage{soul}
%\usepackage{hyperref}
\pgfplotsset{compat=newest}

%\usepackage{amsmath} % Add amssymb if not using Mathtime
\newcommand\numberthis{\addtocounter{equation}{1}\tag{\theequation}}

% Set arrow type
\input{/Users/mqwilber/Dropbox/Documents/Latex/arrowsnew}
\usetikzlibrary{arrows,shapes,backgrounds}
\tikzset{>=latex}

%\usefonttheme{professionalfonts}
\usepackage{fontspec}
\setsansfont[UprightFont={* Light},
             BoldFont={* SemiBold},
             ItalicFont={* Light Italic},
             BoldItalicFont={* SemiBold Italic}]{Gill Sans}

%\setsansfont[UprightFont={* Light}]{Helvetica}
%\setsansfont{Papyrus}
% \setsansfont[Path = /usr/local/texlive/2014/texmf-dist/fonts/truetype/huerta/alegreya/,
% Extension=.ttf
% ]{AlegreyaSC-Regular}

% Set graphics path
\graphicspath{{images/}}


\renewcommand\footnoterule{}
\newcommand{\tc}{\textcolor{red}}
\newcommand{\tb}{\textcolor{blue}}
\newcommand\ig[1]{\includegraphics[width=#1\textwidth]}

\definecolor{dgreen}{rgb}{0.,0.6,0.}
\setbeamertemplate{caption}{\raggedright\insertcaption\par}
% \usepackage[dvipsnames]{xcolor}

% Color of frame title
\setbeamercolor{frametitle}{fg=Black, bg=White}
\setbeamertemplate{frametitle}[two lines]

% Change color of title
\setbeamercolor{title}{fg=Black, bg=White}

% Change color of enumerated list
\setbeamertemplate{enumerate item}{\color{Black}\insertenumlabel.}

\setbeamertemplate{itemize item}{\color{Black}\tikz[baseline=-0.8ex]\draw[black,fill=black] (0,0) circle (.3ex);}
\setbeamertemplate{itemize subitem}{\color{Black}\tikz[baseline=-0.8ex]\draw[black,fill=black] (0,0) circle (.2ex);}

%% Personal commands

% Define frog
\def\cfrog{\includegraphics[width=0.8cm]{/Users/mqwilber/Repos/completed_projects/empirical_taylor_law/docs/presentation/images/frog_sw}}

\def\cfrognew{\includegraphics[width=.15\textwidth]{/Users/mqwilber/Repos/completed_projects/density_dependent_ipm/docs/presentation/images/cfrog}}
\def\cinfrognew{\includegraphics[width=.15\textwidth]{/Users/mqwilber/Repos/completed_projects/density_dependent_ipm/docs/presentation/images/infected_cfrog}}

\def\border{draw, ultra thick, inner sep=.8}

\newcommand{\numfrog}[1]
{   \begin{tikzpicture}
      \node[opacity=0.5] at (0, 0) (frog) {\includegraphics[width=1.4cm]{/Users/mqwilber/Repos/completed_projects/empirical_taylor_law/docs/presentation/images/frog_sw}};
      \node[inner sep=0.8, draw, fill=white] at (0, 0) {\textcolor{black}{#1}};
    \end{tikzpicture}
}

% Footnote without a marker
\newcommand\blfootnote[1]{%
  \begingroup
  \renewcommand\thefootnote{}\footnote{#1}%
  \addtocounter{footnote}{-1}%
  \endgroup
}

\mode<presentation> {

% The Beamer class comes with a number of default slide themes
% which change the colors and layouts of slides. Below this is a list
% of all the themes, uncomment each in turn to see what they look like.

\usetheme{default}
%\usetheme{AnnArbor}
%\usetheme{Antibes}
%\usetheme{Bergen}
%\usetheme{Berkeley}
%\usetheme{Berlin}
%\usetheme{Boadilla}
%\usetheme{CambridgeUS}
%\usetheme{Copenhagen}
%\usetheme{Darmstadt}
%\usetheme{Dresden}
%\usetheme{Frankfurt}
%\usetheme{Goettingen}
%\usetheme{Hannover}
%\usetheme{Ilmenau}
%\usetheme{JuanLesPins}
%\usetheme{Luebeck}
%\usetheme{Madrid}
%\usetheme{Malmoe}
%\usetheme{Marburg}
%\usetheme{Montpellier}
%\usetheme{PaloAlto}
%\usetheme{Pittsburgh}
%\usetheme{Rochester}
%\usetheme{Singapore}
%\usetheme{Szeged}
%\usetheme{Warsaw}

% As well as themes, the Beamer class has a number of color themes
% for any slide theme. Uncomment each of these in turn to see how it
% changes the colors of your current slide theme.

%\usecolortheme{albatross}
%\usecolortheme{beaver}
%\usecolortheme{beetle}
%\usecolortheme{crane}
%\usecolortheme{dolphin}
%\usecolortheme{dove}
%\usecolortheme{fly}
%\usecolortheme{lily}
%\usecolortheme{orchid}
%\usecolortheme{rose}
%\usecolortheme{seagull}
%\usecolortheme{seahorse}
%\usecolortheme{whale}
%\usecolortheme{wolverine}

\setbeamertemplate{footline} % To remove the footer line in all slides uncomment this line
%\setbeamertemplate{footline}[page number] % To replace the footer line in all slides with a simple slide count uncomment this line

\setbeamertemplate{navigation symbols}{} % To remove the navigation symbols from the bottom of all slides uncomment this line
}


% Centers the frametitles of each slide
% \setbeameroption{show notes}
\setbeamertemplate{frametitle}[default][center]

% Adding

\title[Exit seminar]{Spatial scale and resource selection in feral swine} % The
% short
% title appears at the bottom
% of every
% slide, the full title is only on the title page

\author{Lab Meeting}
\institute[UCSB]
{
\smallskip

}
\date{March 4, 2018} % Date, can be changed to a custom date

%\titlegraphic{\includegraphics[scale=0.25]{example_tpl.png}}

\begin{document}

% Remember pictures
\tikzstyle{every picture}+=[remember picture]
\tikzstyle{na} = [baseline=-.5ex]

% \begin{frame}
% \titlepage % Print the title page as the first slide
% \end{frame}

%------------------------------------------------

\begin{frame}[t]
\frametitle{\underline{How do we account for spatial-scale in resource selection?}}
  
  \only<1>{
    \centering
    \ig{0.8}{images/Northrup.png}
    \blfootnote{\hfill \tiny Northrup et al. 2016, \emph{Eco. App.}}
  }

  \only<2-3>{
    \centering

    \begin{tikzpicture}
    
      \node<2-3> at (0, 0) {\ig{0.7}{/Users/mqwilber/Repos/rsf_swine/code/other_scripts/hooten.png}};
      \node<3>[align=left, font=\scriptsize, fill=white] at (0, 1) {1. Location-based covariation \\ 2. \textbf{Gradient-based covariate}};

    \end{tikzpicture}
  }

\end{frame}

%%------------------------------------------------

\begin{frame}[t]
\frametitle{\underline{Example: A pig from Tejon Ranch}}
  
  \only<1-2>{
    \centering 
    \only<1>{\ig{0.7}{images/tejon_ras.pdf}}
    \only<2>{\ig{0.7}{images/tejon_ras_w_pig.pdf}}

    \only<1>{Fruit and nut fields in the proximity of Tejon ranch pigs}
    \only<2>{How is pig movement affected by resource when not in the resource?}
  }

\end{frame}

%%------------------------------------------------

\begin{frame}[t]
\frametitle{\underline{Some options for accounting for spatial scale}}

  \only<1-3>{
  \begin{enumerate}
    \item<1-3> Sum of global distance-to-resource pixel vectors
    \item<1> Distance-to-nearest resource
    \item<1> Some combination of the two?
  \end{enumerate}}

  \only<2>{\centering \ig{0.6}{images/tejon_ras_w_pig.pdf}}

  \only<3>{
    \centering
    \ig{1}{images/tejon_vect.pdf}
  }

  \only<4>{
    \begin{itemize}
      \item Accounts for patch size and distance
      \item Can easily explore functional form of distance decay
        \begin{itemize}
          \item e.g. $\exp(-\gamma D_{ij}^2)$ vs. $\exp(-\gamma D_{ij})$
          \item How does $\gamma$ vary w/ resources? Individuals? Populations?
        \end{itemize}
      \item Computationally infeasible for abundant resources pixels
        \begin{itemize}
          \item Randomly sub-sample resource pixels?
        \end{itemize}
    \end{itemize}
  }

  \only<5>{\centering \ig{1}{images/tejon_samp_ras.png}}
  

\end{frame}


%%------------------------------------------------

\begin{frame}[t]
\frametitle{\underline{Distance-to-nearest resource}}
    
    \only<1>{
    \centering
    \ig{0.8}{images/tejon_shapefile.pdf}}

    \only<2>{
    \centering
    \ig{0.8}{images/tejon_distance.pdf}}

    \only<3>{
      \begin{itemize}
        \item Computationally more feasible
        \item Distance-to-nearest ``X'' is commonly used
        \item Ignores simultaneous influence of multiple patches
        \item Compromise: Distance to all patches weighted by patch area? 
      \end{itemize}
    }

\end{frame}

%%------------------------------------------------

\end{document}

